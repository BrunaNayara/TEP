\section{Implementação}

\begin{frame}[fragile]{Implementação de listas auto-organizáveis}

    \begin{itemize}
        \item As listas auto-organizáveis podem ser implementadas por meio da composição

        \item Assim, as listas auto-organizáveis contém uma lista duplamente encadeada, cuja
            interface fica inacessível ao usuário

        \item Usando as operações de inserção e remoção da lista encadeada é possível
            implementar as formas de organização desejadas

        \item Para manter um registro da frequência pode ser utilizada a classe 
            \code{c}{map} do C++

        \item Para computar a eficiência, basta manter a soma do tamanho da lista a cada 
            busca, e também o número de comparações feitas
    \end{itemize}

\end{frame}

\begin{frame}[fragile]{Exemplo de implementação: Mover para Frente}
    \inputsnippet{c++}{1}{21}{move_to_front.cpp}
\end{frame}

\begin{frame}[fragile]{Exemplo de implementação: Mover para Frente}
    \inputsnippet{c++}{22}{42}{move_to_front.cpp}
\end{frame}

\begin{frame}[fragile]{Exemplo de implementação: Mover para Frente}
    \inputsnippet{c++}{43}{63}{move_to_front.cpp}
\end{frame}

\begin{frame}[fragile]{Exemplo de implementação: Ordenação Estática Ótima}
    \inputsnippet{c++}{1}{21}{optimal.cpp}
\end{frame}

\begin{frame}[fragile]{Exemplo de implementação: Ordenação Estática Ótima}
    \inputsnippet{c++}{22}{42}{optimal.cpp}
\end{frame}

\begin{frame}[fragile]{Exemplo de implementação: Ordenação Estática Ótima}
    \inputsnippet{c++}{44}{64}{optimal.cpp}
\end{frame}
