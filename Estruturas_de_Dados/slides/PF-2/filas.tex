\section{Filas}

\begin{frame}[fragile]{Definição de fila}

    \begin{itemize}
        \item Uma fila é um tipo de dados abstrato cuja interface define que o primeiro elemento
            inserido na pilha é o primeiro a ser removido

        \item Esta estratégia de inserção e remoção é denominada FIFO -- \textit{First In, First
            Out}

        \item De acordo com sua interface, uma fila não permite acesso aleatório ao seus
            elementos (apenas o elemento do topo da fila pode ser acessado)

        \item As operações de inserção e remoção devem ter complexidade $O(1)$

    \end{itemize}

\end{frame}

\begin{frame}[fragile]{Interface de uma fila}

    \begin{table}
        \centering
        \begin{tabularx}{0.95\textwidth}{lcX}
            \toprule
            \textbf{Método} & \textbf{Complexidade} & \textbf{Descrição} \\
            \midrule
            \texttt{clear(F)} & $O(N)$ & Esvazia a fila \texttt{F}, removendo todos os seus elementos \\
            \rowcolor[gray]{0.8}
            \texttt{empty(F)} & $O(1)$ & Verifica se a fila \texttt{F} está vazia ou não \\
            \texttt{push(F, x)} & $O(1)$ & Insere o elemento \texttt{x} no final da fila \texttt{F} \\
            \rowcolor[gray]{0.8}
            \texttt{pop(F)} & $O(1)$ & Remove o elemento que está no início da fila \texttt{F} \\
            \texttt{front(F)} & $O(1)$ & Retorna o elemento que está no início da fila \texttt{F} \\
            \rowcolor[gray]{0.8}
            \texttt{size(F)} & $O(1)$ & Retorna o número de elementos armazenados na fila \texttt{F} \\
            \bottomrule
        \end{tabularx}
    \end{table}
\end{frame}

\begin{frame}[fragile]{Exemplo dos métodos da interface de uma fila}

    \begin{tikzpicture}

        \begin{scope}
            \node[anchor=west] at (0, 6) { \textbf{Método} };
            \node[anchor=west] at (6, 6) { \textbf{Retorno} };

            \node[anchor=west] at (0, 3) { \textbf{Fila} };
            \draw (1, 0) -- (9, 0);

            \node[anchor=west] at (0, 5) { \texttt{empty(F)} };

%            \node (4, 0) rectangle (5, 1);
        \end{scope}

    \end{tikzpicture}

\end{frame}

\begin{frame}[fragile]{Exemplo dos métodos da interface de uma fila}

    \begin{tikzpicture}

        \begin{scope}
            \node[anchor=west] at (0, 6) { \textbf{Método} };
            \node[anchor=west] at (6, 6) { \textbf{Retorno} };
            \node[anchor=west] at (6, 5) { \textbf{\texttt{True}} };

            \node[anchor=west] at (0, 3) { \textbf{Fila} };
            \draw (1, 0) -- (9, 0);

            \node[anchor=west] at (0, 5) { \texttt{empty(F)} };

%            \node (4, 0) rectangle (5, 1);
        \end{scope}

    \end{tikzpicture}

\end{frame}

\begin{frame}[fragile]{Exemplo dos métodos da interface de uma fila}

    \begin{tikzpicture}

        \begin{scope}
            \node[anchor=west] at (0, 6) { \textbf{Método} };
            \node[anchor=west] at (6, 6) { \textbf{Retorno} };
%            \node[anchor=west] at (6, 5) { \textbf{\texttt{True}} };

            \node[anchor=west] at (0, 3) { \textbf{Fila} };
            \draw (1, 0) -- (9, 0);

            \node[anchor=west] at (0, 5) { \texttt{push(F, 5)} };

%            \node (4, 0) rectangle (5, 1);
        \end{scope}

    \end{tikzpicture}

\end{frame}

\begin{frame}[fragile]{Exemplo dos métodos da interface de uma fila}

    \begin{tikzpicture}

        \begin{scope}
            \node[anchor=west] at (0, 6) { \textbf{Método} };
            \node[anchor=west] at (6, 6) { \textbf{Retorno} };
%            \node[anchor=west] at (6, 5) { \textbf{\texttt{True}} };

            \node[anchor=west] at (0, 3) { \textbf{Fila} };
            \draw (1, 0) -- (9, 0);

            \node[anchor=west] at (0, 5) { \texttt{push(F, 5)} };

            \draw (2, 0) rectangle (3, 1);
            \node at (2.5, 0.5) { \texttt{5} };
        \end{scope}

    \end{tikzpicture}

\end{frame}


\begin{frame}[fragile]{Exemplo dos métodos da interface de uma fila}

    \begin{tikzpicture}

        \begin{scope}
            \node[anchor=west] at (0, 6) { \textbf{Método} };
            \node[anchor=west] at (6, 6) { \textbf{Retorno} };
%            \node[anchor=west] at (6, 5) { \textbf{\texttt{True}} };

            \node[anchor=west] at (0, 3) { \textbf{Fila} };
            \draw (1, 0) -- (9, 0);

            \node[anchor=west] at (0, 5) { \texttt{push(F, 11)} };

            \draw (2, 0) rectangle (3, 1);
            \node at (2.5, 0.5) { \texttt{5} };
        \end{scope}

    \end{tikzpicture}

\end{frame}


\begin{frame}[fragile]{Exemplo dos métodos da interface de uma fila}

    \begin{tikzpicture}

        \begin{scope}
            \node[anchor=west] at (0, 6) { \textbf{Método} };
            \node[anchor=west] at (6, 6) { \textbf{Retorno} };
%            \node[anchor=west] at (6, 5) { \textbf{\texttt{True}} };

            \node[anchor=west] at (0, 3) { \textbf{Fila} };
            \draw (1, 0) -- (9, 0);

            \node[anchor=west] at (0, 5) { \texttt{push(F, 11)} };

            \draw (2, 0) rectangle (3, 1);
            \node at (2.5, 0.5) { \texttt{5} };

            \draw (4, 0) rectangle (5, 1);
            \node at (4.5, 0.5) { \texttt{11} };
        \end{scope}

    \end{tikzpicture}

\end{frame}


\begin{frame}[fragile]{Exemplo dos métodos da interface de uma fila}

    \begin{tikzpicture}

        \begin{scope}
            \node[anchor=west] at (0, 6) { \textbf{Método} };
            \node[anchor=west] at (6, 6) { \textbf{Retorno} };
%            \node[anchor=west] at (6, 5) { \textbf{\texttt{True}} };

            \node[anchor=west] at (0, 3) { \textbf{Fila} };
            \draw (1, 0) -- (9, 0);

            \node[anchor=west] at (0, 5) { \texttt{push(F, 7)} };

            \draw (2, 0) rectangle (3, 1);
            \node at (2.5, 0.5) { \texttt{5} };

            \draw (4, 0) rectangle (5, 1);
            \node at (4.5, 0.5) { \texttt{11} };
        \end{scope}

    \end{tikzpicture}

\end{frame}


\begin{frame}[fragile]{Exemplo dos métodos da interface de uma fila}

    \begin{tikzpicture}

        \begin{scope}
            \node[anchor=west] at (0, 6) { \textbf{Método} };
            \node[anchor=west] at (6, 6) { \textbf{Retorno} };
%            \node[anchor=west] at (6, 5) { \textbf{\texttt{True}} };

            \node[anchor=west] at (0, 3) { \textbf{Fila} };
            \draw (1, 0) -- (9, 0);

            \node[anchor=west] at (0, 5) { \texttt{push(F, 7)} };

            \draw (2, 0) rectangle (3, 1);
            \node at (2.5, 0.5) { \texttt{5} };

            \draw (4, 0) rectangle (5, 1);
            \node at (4.5, 0.5) { \texttt{11} };

            \draw (6, 0) rectangle (7, 1);
            \node at (6.5, 0.5) { \texttt{7} };
        \end{scope}

    \end{tikzpicture}

\end{frame}


\begin{frame}[fragile]{Exemplo dos métodos da interface de uma fila}

    \begin{tikzpicture}

        \begin{scope}
            \node[anchor=west] at (0, 6) { \textbf{Método} };
            \node[anchor=west] at (6, 6) { \textbf{Retorno} };
%            \node[anchor=west] at (6, 5) { \textbf{\texttt{True}} };

            \node[anchor=west] at (0, 3) { \textbf{Fila} };
            \draw (1, 0) -- (9, 0);

            \node[anchor=west] at (0, 5) { \texttt{pop(F)} };

            \draw (2, 0) rectangle (3, 1);
            \node at (2.5, 0.5) { \texttt{5} };

            \draw (4, 0) rectangle (5, 1);
            \node at (4.5, 0.5) { \texttt{11} };

            \draw (6, 0) rectangle (7, 1);
            \node at (6.5, 0.5) { \texttt{7} };
        \end{scope}

    \end{tikzpicture}

\end{frame}


\begin{frame}[fragile]{Exemplo dos métodos da interface de uma fila}

    \begin{tikzpicture}

        \begin{scope}
            \node[anchor=west] at (0, 6) { \textbf{Método} };
            \node[anchor=west] at (6, 6) { \textbf{Retorno} };
%            \node[anchor=west] at (6, 5) { \textbf{\texttt{True}} };

            \node[anchor=west] at (0, 3) { \textbf{Fila} };
            \draw (1, 0) -- (9, 0);

            \node[anchor=west] at (0, 5) { \texttt{pop(F)} };

            \draw (2, 0) rectangle (3, 1);
            \node at (2.5, 0.5) { \texttt{11} };

            \draw (4, 0) rectangle (5, 1);
            \node at (4.5, 0.5) { \texttt{7} };
        \end{scope}

    \end{tikzpicture}

\end{frame}


\begin{frame}[fragile]{Exemplo dos métodos da interface de uma fila}

    \begin{tikzpicture}

        \begin{scope}
            \node[anchor=west] at (0, 6) { \textbf{Método} };
            \node[anchor=west] at (6, 6) { \textbf{Retorno} };
%            \node[anchor=west] at (6, 5) { \textbf{\texttt{True}} };

            \node[anchor=west] at (0, 3) { \textbf{Fila} };
            \draw (1, 0) -- (9, 0);

            \node[anchor=west] at (0, 5) { \texttt{size(F)} };

            \draw (2, 0) rectangle (3, 1);
            \node at (2.5, 0.5) { \texttt{11} };

            \draw (4, 0) rectangle (5, 1);
            \node at (4.5, 0.5) { \texttt{7} };
        \end{scope}

    \end{tikzpicture}

\end{frame}


\begin{frame}[fragile]{Exemplo dos métodos da interface de uma fila}

    \begin{tikzpicture}

        \begin{scope}
            \node[anchor=west] at (0, 6) { \textbf{Método} };
            \node[anchor=west] at (6, 6) { \textbf{Retorno} };
            \node[anchor=west] at (6, 5) { \texttt{2} };

            \node[anchor=west] at (0, 3) { \textbf{Fila} };
            \draw (1, 0) -- (9, 0);

            \node[anchor=west] at (0, 5) { \texttt{size(F)} };

            \draw (2, 0) rectangle (3, 1);
            \node at (2.5, 0.5) { \texttt{11} };

            \draw (4, 0) rectangle (5, 1);
            \node at (4.5, 0.5) { \texttt{7} };
        \end{scope}

    \end{tikzpicture}

\end{frame}


\begin{frame}[fragile]{Exemplo dos métodos da interface de uma fila}

    \begin{tikzpicture}

        \begin{scope}
            \node[anchor=west] at (0, 6) { \textbf{Método} };
            \node[anchor=west] at (6, 6) { \textbf{Retorno} };
%            \node[anchor=west] at (6, 5) { \texttt{2} };

            \node[anchor=west] at (0, 3) { \textbf{Fila} };
            \draw (1, 0) -- (9, 0);

            \node[anchor=west] at (0, 5) { \texttt{front(F)} };

            \draw (2, 0) rectangle (3, 1);
            \node at (2.5, 0.5) { \texttt{11} };

            \draw (4, 0) rectangle (5, 1);
            \node at (4.5, 0.5) { \texttt{7} };
        \end{scope}

    \end{tikzpicture}

\end{frame}


\begin{frame}[fragile]{Exemplo dos métodos da interface de uma fila}

    \begin{tikzpicture}

        \begin{scope}
            \node[anchor=west] at (0, 6) { \textbf{Método} };
            \node[anchor=west] at (6, 6) { \textbf{Retorno} };
            \node[anchor=west] at (6, 5) { \texttt{11} };

            \node[anchor=west] at (0, 3) { \textbf{Fila} };
            \draw (1, 0) -- (9, 0);

            \node[anchor=west] at (0, 5) { \texttt{front(F)} };

            \draw (2, 0) rectangle (3, 1);
            \node at (2.5, 0.5) { \texttt{11} };

            \draw (4, 0) rectangle (5, 1);
            \node at (4.5, 0.5) { \texttt{7} };
        \end{scope}

    \end{tikzpicture}

\end{frame}

\begin{frame}[fragile]{Implementação de uma fila}

    \begin{itemize}
        \item Como uma pilha é um tipo de dados abstrato, ela não impõe nenhuma restrição
            quanto à sua implementação

        \item É possível implementar uma pilha por composição, usando listas encadeadas ou um
            deque

        \item A estratégia FIFO precisa de operações de inserção e remoção eficientes nos
            dois extremos do contêiner, o que inviabiliza o uso do \code{c}{vector} e da
            \code{c}{forward_list}

        \item Se há uma estimativa do tamanho máximo de elementos a serem inseridos na fila,
            é possível usar um \textit{array} estático e o mesmo princípio de uma lista circular 
            para implementar uma fila
    \end{itemize}

\end{frame}

\begin{frame}[fragile]{Exemplo de implementação de fila em C++}
    \inputsnippet{c++}{1}{21}{queue.cpp}
\end{frame}

\begin{frame}[fragile]{Exemplo de implementação de fila em C++}
    \inputsnippet{c++}{22}{42}{queue.cpp}
\end{frame}

\begin{frame}[fragile]{Exemplo de implementação de fila em C++}
    \inputsnippet{c++}{44}{64}{queue.cpp}
\end{frame}

\begin{frame}[fragile]{Filas em C++}

    \begin{itemize}
        \item A STL do C++ oferece uma implementação de fila: a classe \code{c}{queue}

        \item Assim como no caso das pilhas, o contêiner usado na composição é, por padrão,
            o \code{c}{deque}

        \item Este contêiner pode ser substituido por qualquer contêiner que contenha os métodos
            \code{c}{pop_front(), push_back()} e \code{c}{size()}, dentre outros

        \item O método \code{c}{swap()} também está disponível
    \end{itemize}

\end{frame}
