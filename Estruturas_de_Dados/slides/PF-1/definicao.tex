\section{Definição}

\begin{frame}[fragile]{Definição de pilha}

    \begin{itemize}
        \item Uma pilha é um tipo de dados abstrato cuja interface define que o último elemento
            inserido na pilha é o primeiro a ser removido

        \item Esta estratégia de inserção e remoção é denominada LIFO -- \textit{Last In, First
            Out}

        \item De acordo com sua interface, uma pilha não permite acesso aleatório ao seus
            elementos (apenas o elemento do topo da pilha pode ser acessado)

        \item As operações de inserção e remoção devem ter complexidade $O(1)$

    \end{itemize}

\end{frame}

\begin{frame}[fragile]{Interface de uma pilha}

    \begin{table}
        \centering
        \begin{tabularx}{0.95\textwidth}{lcX}
            \toprule
            \textbf{Método} & \textbf{Complexidade} & \textbf{Descrição} \\
            \midrule
            \texttt{clear(P)} & $O(N)$ & Esvazia a pilha \texttt{P}, removendo todos os seus elementos \\
            \rowcolor[gray]{0.8}
            \texttt{empty(P)} & $O(1)$ & Verifica se a pilha \texttt{P} está vazia ou não \\
            \texttt{push(P, x)} & $O(1)$ & Insere o elemento \texttt{x} no topo da pilha \texttt{P} \\
            \rowcolor[gray]{0.8}
            \texttt{pop(P)} & $O(1)$ & Remove o elemento que está no topo da pilha \texttt{P} \\
            \texttt{top(P)} & $O(1)$ & Retorna o elemento que está no topo da pilha \texttt{P} \\
            \rowcolor[gray]{0.8}
            \texttt{size(P)} & $O(1)$ & Retorna o número de elementos armazenados na pilha \texttt{P} \\
            \bottomrule
        \end{tabularx}
    \end{table}
\end{frame}

\begin{frame}[fragile]{Exemplo dos métodos da interface de uma pilha}

    \begin{tikzpicture}

        \begin{scope}
            \node[anchor=west] at (0, 6) { \textbf{Método} };
            \node[anchor=west] at (6, 6) { \textbf{Retorno} };

            \node[anchor=west] at (0, 4) { \textbf{Pilha} };
            \draw (2, 0) -- (7, 0);

            \node[anchor=west] at (0, 5) { \texttt{empty(P)} };

%            \node (4, 0) rectangle (5, 1);
        \end{scope}

    \end{tikzpicture}

\end{frame}

\begin{frame}[fragile]{Exemplo dos métodos da interface de uma pilha}

    \begin{tikzpicture}

        \begin{scope}
            \node[anchor=west] at (0, 6) { \textbf{Método} };
            \node[anchor=west] at (6, 6) { \textbf{Retorno} };

            \node[anchor=west] at (0, 4) { \textbf{Pilha} };
            \draw (2, 0) -- (7, 0);

            \node[anchor=west] at (0, 5) { \texttt{empty(P)} };
            \node[anchor=west] at (6, 5) { \textbf{\texttt{True}} };

%            \node (4, 0) rectangle (5, 1);
        \end{scope}

    \end{tikzpicture}

\end{frame}

\begin{frame}[fragile]{Exemplo dos métodos da interface de uma pilha}

    \begin{tikzpicture}

        \begin{scope}
            \node[anchor=west] at (0, 6) { \textbf{Método} };
            \node[anchor=west] at (6, 6) { \textbf{Retorno} };

            \node[anchor=west] at (0, 4) { \textbf{Pilha} };
            \draw (2, 0) -- (7, 0);

            \node[anchor=west] at (0, 5) { \texttt{push(P, 12)} };
%            \node[anchor=west] at (6, 5) { \textbf{\texttt{True}} };

%            \node (4, 0) rectangle (5, 1);
        \end{scope}

    \end{tikzpicture}

\end{frame}

\begin{frame}[fragile]{Exemplo dos métodos da interface de uma pilha}

    \begin{tikzpicture}

        \begin{scope}
            \node[anchor=west] at (0, 6) { \textbf{Método} };
            \node[anchor=west] at (6, 6) { \textbf{Retorno} };

            \node[anchor=west] at (0, 4) { \textbf{Pilha} };
            \draw (2, 0) -- (7, 0);

            \node[anchor=west] at (0, 5) { \texttt{push(P, 12)} };
%            \node[anchor=west] at (6, 5) { \textbf{\texttt{True}} };

            \draw (4, 0) rectangle (5, 1);
            \node at (4.5, 0.5) { \texttt{12} };
        \end{scope}

    \end{tikzpicture}

\end{frame}

\begin{frame}[fragile]{Exemplo dos métodos da interface de uma pilha}

    \begin{tikzpicture}

        \begin{scope}
            \node[anchor=west] at (0, 6) { \textbf{Método} };
            \node[anchor=west] at (6, 6) { \textbf{Retorno} };

            \node[anchor=west] at (0, 4) { \textbf{Pilha} };
            \draw (2, 0) -- (7, 0);

            \node[anchor=west] at (0, 5) { \texttt{push(P, -3)} };
%            \node[anchor=west] at (6, 5) { \textbf{\texttt{True}} };

            \draw (4, 0) rectangle (5, 1);
            \node at (4.5, 0.5) { \texttt{12} };
        \end{scope}

    \end{tikzpicture}

\end{frame}

\begin{frame}[fragile]{Exemplo dos métodos da interface de uma pilha}

    \begin{tikzpicture}

        \begin{scope}
            \node[anchor=west] at (0, 6) { \textbf{Método} };
            \node[anchor=west] at (6, 6) { \textbf{Retorno} };

            \node[anchor=west] at (0, 4) { \textbf{Pilha} };
            \draw (2, 0) -- (7, 0);

            \node[anchor=west] at (0, 5) { \texttt{push(P, -3)} };
%            \node[anchor=west] at (6, 5) { \textbf{\texttt{True}} };

            \draw (4, 0) rectangle (5, 1);
            \node at (4.5, 0.5) { \texttt{12} };

            \draw (4, 1) rectangle (5, 2);
            \node at (4.5, 1.5) { \texttt{-3} };
        \end{scope}

    \end{tikzpicture}

\end{frame}

\begin{frame}[fragile]{Exemplo dos métodos da interface de uma pilha}

    \begin{tikzpicture}

        \begin{scope}
            \node[anchor=west] at (0, 6) { \textbf{Método} };
            \node[anchor=west] at (6, 6) { \textbf{Retorno} };

            \node[anchor=west] at (0, 4) { \textbf{Pilha} };
            \draw (2, 0) -- (7, 0);

            \node[anchor=west] at (0, 5) { \texttt{pop(P)} };
%            \node[anchor=west] at (6, 5) { \textbf{\texttt{True}} };

            \draw (4, 0) rectangle (5, 1);
            \node at (4.5, 0.5) { \texttt{12} };

            \draw (4, 1) rectangle (5, 2);
            \node at (4.5, 1.5) { \texttt{-3} };
        \end{scope}

    \end{tikzpicture}

\end{frame}

\begin{frame}[fragile]{Exemplo dos métodos da interface de uma pilha}

    \begin{tikzpicture}

        \begin{scope}
            \node[anchor=west] at (0, 6) { \textbf{Método} };
            \node[anchor=west] at (6, 6) { \textbf{Retorno} };

            \node[anchor=west] at (0, 4) { \textbf{Pilha} };
            \draw (2, 0) -- (7, 0);

            \node[anchor=west] at (0, 5) { \texttt{pop(P)} };
            %\node[anchor=west] at (6, 5) { \texttt{-3} };

            \draw (4, 0) rectangle (5, 1);
            \node at (4.5, 0.5) { \texttt{12} };

%            \draw (4, 1) rectangle (5, 2);
%            \node at (4.5, 1.5) { \texttt{-3} };
        \end{scope}

    \end{tikzpicture}

\end{frame}

\begin{frame}[fragile]{Exemplo dos métodos da interface de uma pilha}

    \begin{tikzpicture}

        \begin{scope}
            \node[anchor=west] at (0, 6) { \textbf{Método} };
            \node[anchor=west] at (6, 6) { \textbf{Retorno} };

            \node[anchor=west] at (0, 4) { \textbf{Pilha} };
            \draw (2, 0) -- (7, 0);

            \node[anchor=west] at (0, 5) { \texttt{push(P, 5)} };
%            \node[anchor=west] at (6, 5) { \texttt{-3} };

            \draw (4, 0) rectangle (5, 1);
            \node at (4.5, 0.5) { \texttt{12} };

%            \draw (4, 1) rectangle (5, 2);
%            \node at (4.5, 1.5) { \texttt{-3} };
        \end{scope}

    \end{tikzpicture}

\end{frame}

\begin{frame}[fragile]{Exemplo dos métodos da interface de uma pilha}

    \begin{tikzpicture}

        \begin{scope}
            \node[anchor=west] at (0, 6) { \textbf{Método} };
            \node[anchor=west] at (6, 6) { \textbf{Retorno} };

            \node[anchor=west] at (0, 4) { \textbf{Pilha} };
            \draw (2, 0) -- (7, 0);

            \node[anchor=west] at (0, 5) { \texttt{push(P, 5)} };
%            \node[anchor=west] at (6, 5) { \texttt{-3} };

            \draw (4, 0) rectangle (5, 1);
            \node at (4.5, 0.5) { \texttt{12} };

            \draw (4, 1) rectangle (5, 2);
            \node at (4.5, 1.5) { \texttt{5} };
        \end{scope}

    \end{tikzpicture}

\end{frame}

\begin{frame}[fragile]{Exemplo dos métodos da interface de uma pilha}

    \begin{tikzpicture}

        \begin{scope}
            \node[anchor=west] at (0, 6) { \textbf{Método} };
            \node[anchor=west] at (6, 6) { \textbf{Retorno} };

            \node[anchor=west] at (0, 4) { \textbf{Pilha} };
            \draw (2, 0) -- (7, 0);

            \node[anchor=west] at (0, 5) { \texttt{push(P, 47)} };
%            \node[anchor=west] at (6, 5) { \texttt{-3} };

            \draw (4, 0) rectangle (5, 1);
            \node at (4.5, 0.5) { \texttt{12} };

            \draw (4, 1) rectangle (5, 2);
            \node at (4.5, 1.5) { \texttt{5} };
        \end{scope}

    \end{tikzpicture}

\end{frame}

\begin{frame}[fragile]{Exemplo dos métodos da interface de uma pilha}

    \begin{tikzpicture}

        \begin{scope}
            \node[anchor=west] at (0, 6) { \textbf{Método} };
            \node[anchor=west] at (6, 6) { \textbf{Retorno} };

            \node[anchor=west] at (0, 4) { \textbf{Pilha} };
            \draw (2, 0) -- (7, 0);

            \node[anchor=west] at (0, 5) { \texttt{push(P, 47)} };
%            \node[anchor=west] at (6, 5) { \texttt{-3} };

            \draw (4, 0) rectangle (5, 1);
            \node at (4.5, 0.5) { \texttt{12} };

            \draw (4, 1) rectangle (5, 2);
            \node at (4.5, 1.5) { \texttt{5} };

            \draw (4, 2) rectangle (5, 3);
            \node at (4.5, 2.5) { \texttt{47} };
        \end{scope}

    \end{tikzpicture}

\end{frame}

\begin{frame}[fragile]{Exemplo dos métodos da interface de uma pilha}

    \begin{tikzpicture}

        \begin{scope}
            \node[anchor=west] at (0, 6) { \textbf{Método} };
            \node[anchor=west] at (6, 6) { \textbf{Retorno} };

            \node[anchor=west] at (0, 4) { \textbf{Pilha} };
            \draw (2, 0) -- (7, 0);

            \node[anchor=west] at (0, 5) { \texttt{top(P)} };
%            \node[anchor=west] at (6, 5) { \texttt{-3} };

            \draw (4, 0) rectangle (5, 1);
            \node at (4.5, 0.5) { \texttt{12} };

            \draw (4, 1) rectangle (5, 2);
            \node at (4.5, 1.5) { \texttt{5} };

            \draw (4, 2) rectangle (5, 3);
            \node at (4.5, 2.5) { \texttt{47} };
        \end{scope}

    \end{tikzpicture}

\end{frame}

\begin{frame}[fragile]{Exemplo dos métodos da interface de uma pilha}

    \begin{tikzpicture}

        \begin{scope}
            \node[anchor=west] at (0, 6) { \textbf{Método} };
            \node[anchor=west] at (6, 6) { \textbf{Retorno} };

            \node[anchor=west] at (0, 4) { \textbf{Pilha} };
            \draw (2, 0) -- (7, 0);

            \node[anchor=west] at (0, 5) { \texttt{top(P)} };
            \node[anchor=west] at (6, 5) { \texttt{47} };

            \draw (4, 0) rectangle (5, 1);
            \node at (4.5, 0.5) { \texttt{12} };

            \draw (4, 1) rectangle (5, 2);
            \node at (4.5, 1.5) { \texttt{5} };

            \draw (4, 2) rectangle (5, 3);
            \node at (4.5, 2.5) { \texttt{47} };
        \end{scope}

    \end{tikzpicture}

\end{frame}

\begin{frame}[fragile]{Exemplo dos métodos da interface de uma pilha}

    \begin{tikzpicture}

        \begin{scope}
            \node[anchor=west] at (0, 6) { \textbf{Método} };
            \node[anchor=west] at (6, 6) { \textbf{Retorno} };

            \node[anchor=west] at (0, 4) { \textbf{Pilha} };
            \draw (2, 0) -- (7, 0);

            \node[anchor=west] at (0, 5) { \texttt{size(P)} };
%            \node[anchor=west] at (6, 5) { \texttt{47} };

            \draw (4, 0) rectangle (5, 1);
            \node at (4.5, 0.5) { \texttt{12} };

            \draw (4, 1) rectangle (5, 2);
            \node at (4.5, 1.5) { \texttt{5} };

            \draw (4, 2) rectangle (5, 3);
            \node at (4.5, 2.5) { \texttt{47} };
        \end{scope}

    \end{tikzpicture}

\end{frame}

\begin{frame}[fragile]{Exemplo dos métodos da interface de uma pilha}

    \begin{tikzpicture}

        \begin{scope}
            \node[anchor=west] at (0, 6) { \textbf{Método} };
            \node[anchor=west] at (6, 6) { \textbf{Retorno} };

            \node[anchor=west] at (0, 4) { \textbf{Pilha} };
            \draw (2, 0) -- (7, 0);

            \node[anchor=west] at (0, 5) { \texttt{size(P)} };
            \node[anchor=west] at (6, 5) { \texttt{3} };

            \draw (4, 0) rectangle (5, 1);
            \node at (4.5, 0.5) { \texttt{12} };

            \draw (4, 1) rectangle (5, 2);
            \node at (4.5, 1.5) { \texttt{5} };

            \draw (4, 2) rectangle (5, 3);
            \node at (4.5, 2.5) { \texttt{47} };
        \end{scope}

    \end{tikzpicture}

\end{frame}

\begin{frame}[fragile]{Exemplo dos métodos da interface de uma pilha}

    \begin{tikzpicture}

        \begin{scope}
            \node[anchor=west] at (0, 6) { \textbf{Método} };
            \node[anchor=west] at (6, 6) { \textbf{Retorno} };

            \node[anchor=west] at (0, 4) { \textbf{Pilha} };
            \draw (2, 0) -- (7, 0);

            \node[anchor=west] at (0, 5) { \texttt{clear(P)} };
%            \node[anchor=west] at (6, 5) { \texttt{3} };

            \draw (4, 0) rectangle (5, 1);
            \node at (4.5, 0.5) { \texttt{12} };

            \draw (4, 1) rectangle (5, 2);
            \node at (4.5, 1.5) { \texttt{5} };

            \draw (4, 2) rectangle (5, 3);
            \node at (4.5, 2.5) { \texttt{47} };
        \end{scope}

    \end{tikzpicture}

\end{frame}

\begin{frame}[fragile]{Exemplo dos métodos da interface de uma pilha}

    \begin{tikzpicture}

        \begin{scope}
            \node[anchor=west] at (0, 6) { \textbf{Método} };
            \node[anchor=west] at (6, 6) { \textbf{Retorno} };

            \node[anchor=west] at (0, 4) { \textbf{Pilha} };
            \draw (2, 0) -- (7, 0);

            \node[anchor=west] at (0, 5) { \texttt{clear(P)} };
%            \node[anchor=west] at (6, 5) { \texttt{3} };

       \end{scope}

    \end{tikzpicture}

\end{frame}
