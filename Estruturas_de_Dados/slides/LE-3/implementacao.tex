\section{Implementação}

\begin{frame}[fragile]{Implementação de uma lista circular}

    \begin{itemize}
        \item Uma lista circular pode ser implementada a partir de adaptações pontuais
            nas implementações das listas encadeadas ou duplamente encadeadas

        \item A lista que servirá como ponto de partida deve ser escolhida de acordo com
            a memória disponível ou dos sentidos de travessia desejados

        \item Deve ser mantida a invariante que o último elemento aponta para o primeiro
            após cada operação (e que o primeiro aponta para o último, no caso das 
            listas duplamente encadeadas)

        \item É preciso oferecer um iterador circular, que permita a travessia circular dos
            elementos da lista
    \end{itemize}

\end{frame}

\begin{frame}[fragile]{Exemplo de implementação de uma lista circular}
    \inputsnippet{cpp}{1}{21}{circular_list.h}
\end{frame}

\begin{frame}[fragile]{Exemplo de implementação de uma lista circular}
    \inputsnippet{cpp}{23}{43}{circular_list.h}
\end{frame}

\begin{frame}[fragile]{Exemplo de implementação de uma lista circular}
    \inputsnippet{cpp}{44}{64}{circular_list.h}
\end{frame}

\begin{frame}[fragile]{Exemplo de implementação de uma lista circular}
    \inputsnippet{cpp}{65}{85}{circular_list.h}
\end{frame}

\begin{frame}[fragile]{Exemplo de implementação de uma lista circular}
    \inputsnippet{cpp}{87}{107}{circular_list.h}
\end{frame}

\begin{frame}[fragile]{Exemplo de implementação de uma lista circular}
    \inputsnippet{cpp}{108}{128}{circular_list.h}
\end{frame}
