\section{Tipos de dados abstratos}

\begin{frame}[fragile]{Tipos de dados abstratos}

	\begin{itemize}
		\item Os tipos de dados abstratos (\textit{Abstract Data Type -- ADT}) são definidos pelas 
		operações que agem sobre estes dados, e não pela implementação destas operações

		\item Conhecidas as operações, é possível escolher a implementação mais eficiente para 
        cada caso

		\item Em alguns casos também é preciso determinar as restrições que cada operação possui, 
        sejam restrições de comportamento ou de ordem de complexidade
		
	\end{itemize}
 
\end{frame}  

\begin{frame}[fragile]{Operações típicas de tipos de dados abstratos}

	\begin{table}
        \centering
        \begin{tabularx}{\textwidth}{lX}
        \toprule
        \textbf{Operação} & \textbf{Descrição} \\
        \midrule
		\rawcode{create()} & cria uma nova instância \rawcode{S} \\
        \rowcolor[gray]{0.9}
		\rawcode{initialize(S)} & prepara a instância \rawcode{S} para o seu estado inicial, 
        permitindo seu uso em operações subsequentes \\
		\rawcode{free(S)} & libera a memória utilizada pela instância \rawcode{S} \\
        \rowcolor[gray]{0.9}
		\rawcode{empty(S)} & retorna verdadeiro se a instância \rawcode{S} está vazia\\
		\rawcode{size(S)} & lista o número de elementos contidos em \rawcode{S} \\
        \rowcolor[gray]{0.9}
		\rawcode{compare(S, T)} & retorna verdadeiro se as instâncias \rawcode{S} e 
        \rawcode{T} tem os mesmos elementos, na mesma ordem \\
		\rawcode{print(S)} & produz uma representação visual da instância \rawcode{S} \\
        \rowcolor[gray]{0.9}
		\rawcode{copy(S, T)} & faz com que a instância \rawcode{S} fique com o mesmo estado
        que a instância \rawcode{T} \\
        \rawcode{clone(S)} & cria uma {nova} instância \rawcode{T} com o mesmo estado de \rawcode{S} \\
        \bottomrule
        \end{tabularx}
    \end{table}

\end{frame}

\begin{frame}[fragile]{Contêiners}

	\begin{itemize}

		\item Contêiners são tipos de dados abstratos que representam coleções de outros objetos

		\item Além das operações \code{c}{create} e \code{c}{size}, os contêiners devem fornecer 
        operações para:
		\begin{itemize}
			\item remover todos os elementos de uma só vez (\code{c}{clear})
			\item inserir novos elementos (\code{c}{push})
			\item remover elementos (\code{c}{pop})
			\item acesso aos elementos armazenados (\code{c}{element})
		\end{itemize}

		\item A classe \code{c}{vector} de C++ é um contêiner

		\item É possível criar um contêiner semelhante em C, implementando as operações listadas 
		anteriormente

		\item Para tal, é preciso definir uma estrutura que contenha as variáveis necessárias para 
        a implementação das funções que agirão sobre esta estrutura
	\end{itemize}

\end{frame}

\begin{frame}[fragile]{Exemplo de uso da classe \texttt{vector}}
    \inputcode{cpp}{vec.cpp}
\end{frame}
