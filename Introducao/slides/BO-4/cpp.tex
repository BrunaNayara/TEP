\section{Ordenação em C++}

\begin{frame}[fragile]{Algoritmos de ordenação em C++}

    \begin{itemize}
        \item A biblioteca \code{c++}{algorithm} da linguagem C++ oferece 4 algoritmos de
            ordenação: \code{c++}{sort(), stable_sort(), partial_sort()} e \code{c++}{nth_element()}

        \item As interfaces destas funções são semelhantes, e a diferença entre elas está 
            no comportamento de cada uma

        \item Se indicada, a comparação entre dois elementos a serem ordenados é feita por 
            uma função de comparação binária $f(a, b)$, que deve retornar verdadeiro se
            $a$ precede $b$ na ordenação, e falso, em caso contrário

        \item Se a função de comparação for omitida, a comparação será feita através do
            operador \code{c++}{<}

        \item A complexidade média dos três primeiros algoritmos é $O(N\log N)$, e a função
            \code{c++}{nth_element} tem complexidade média $O(N)$
    \end{itemize}

\end{frame}


\begin{frame}[fragile]{Função \texttt{sort()}}

    \begin{itemize}
        \item A função \code{c++}{sort()} implementa um algoritmo de ordenação instável,
            \textit{in-place}, com complexidade média $O(N\log N)$

        \item Duas assinaturas possíveis da função \code{c++}{sort()} são
            \inputsyntax{c++}{sort.st}

        \item Ao final da execução, o intervalo \code{c++}{[first, last)} estará ordenado, de
            acordo com o operator \code{c++}{<} ou o comparador \code{c++}{comp}, respectivamente

        \item A segunda assinatura permite uma maior flexibilidade, pois permite a customização
            do critério de comparação

        \item Se o critério é apenas a substituição do operador \code{c++}{<} pelo operador
            \code{c++}{>}, basta usar a estrutura \code{c++}{greater} da biblioteca padrão
            do C++

    \end{itemize}

\end{frame}

\begin{frame}[fragile]{Exemplo de uso da função \texttt{sort}}
    \inputcode{c++}{sort.cpp}
\end{frame}

\begin{frame}[fragile]{Comparadores}

    \begin{itemize}
        \item O comparador pode ser implementado de 3 maneiras: como uma função binária,
            como uma função binária anônima (\textit{lambda}) ou como uma estrutura com
            o operador booleano binário \code{c++}{()}

        \item A primeira forma tem como vantagem a familiaridade da declaração de funções,
            mas polui o código com uma função de uso específico

        \item A terceira forma traz o mesmo problema, mas o uso da estrutura 
            reduz a poluição do espaço de nomes

        \item A segunda forma evita os problemas de nomes por usar uma função anônima, e a
            declaração na chamada do algoritmo de ordenação reduz a distância entre declaração e uso

        \item Contudo, tem sintaxe menos intuititva e pode levar à duplicação de código, caso
            precise ser utilizada mais de uma vez
        
        \item Alternativamente, pode-se implementar o operador \code{c++}{<} na própria 
            classe dos objetos a serem comparados e usar a primeira assinatura
    \end{itemize}

\end{frame}

\begin{frame}[fragile]{Exemplo de uso de comparadores}
    \inputsnippet{c++}{1}{21}{comp.cpp}
\end{frame}

\begin{frame}[fragile]{Exemplo de uso de comparadores}
    \inputsnippet{c++}{22}{42}{comp.cpp}
\end{frame}

\begin{frame}[fragile]{Exemplo de uso de comparadores}
    \inputsnippet{c++}{43}{63}{comp.cpp}
\end{frame}

\begin{frame}[fragile]{Função \texttt{stable\_sort()}}

    \begin{itemize}

        \item A função \code{c++}{stable_sort()} implementa um algoritmo de ordenação estável,
            \textit{in-place}, com complexidade média $O(N\log N)$
        
        \item Duas assinaturas possíveis da função \code{c++}{stable_sort()} são
            \inputsyntax{c++}{stable_sort.st}

        \item Ao contrário da função \code{c++}{sort()}, a função \code{c++}{stable_sort()}
            preserva a ordem relativa dos elementos considerados iguais

        \item É possível transformar qualquer algoritmo de ordenação instável em um algoritmo
            estável, adicionado-se a cada elemento da sequência um identificador inteiro 
            único de posição 

        \item Contudo, esta adaptação tem custo de memória $O(N)$, além de implicar ou na 
            alteração dos elementos da sequência ou no uso de pares ao invés dos elementos
    \end{itemize}

\end{frame}

\begin{frame}[fragile]{Exemplo de uso da função \texttt{stable\_sort()}}
    \inputcode{c++}{stable.cpp}
\end{frame}

\begin{frame}[fragile]{Título}

    \begin{itemize}

        \item A função \code{c++}{partial_sort()} é semelhante à rotina de pivoteamente do
            \textit{quicksort}, ordenando os elementos que ficam à esquerda da posição 
            indicada e deixando os demais elementos em uma ordem não especificada

        \item Uma função semelhante à \code{c++}{partial_sort()} é a função 
            \code{c++}{nth_element()}, que posiciona corretamente apenas o elemento que
            ocuparia a $n$-ésima posição do vetor, caso estivesse ordenado, com complexidade
            $O(N)$
    \end{itemize}

\end{frame}
