\section{Ordenação em C}

\begin{frame}[fragile]{Quicksort}

    \begin{itemize}
        \item A biblioteca \code{c}{stdlib.h} da linguagem C contém a função 
            \code{c}{qsort()}, a qual implementa o algoritmo \textit{quicksort}

        \item A assinatura da função \code{c}{qsort()} é
            \inputsyntax{c}{qsort.st}

        \item O parâmetro \code{c}{base} é o ponteiro para o primeiro elemento do vetor a 
            ser ordenado

        \item Como o \textit{quicksort} é um algoritmo de ordenação \textit{in-place}, o vetor
            apontado por \code{c}{base} será modificado pela função \code{c}{qsort()}

        \item O parâmetro \code{c}{nmemb} deve indicar o número de elementos a serem ordenados

        \item O parâmetro \code{c}{size} indica o tamanho de um elemento, em \textit{bytes}
    \end{itemize}

\end{frame}

\begin{frame}[fragile]{Função de comparação}

    \begin{itemize}
        \item O último parâmetro da função \code{c}{qsort} é um ponteiro para a função de
            comparação \code{c}{compar}

        \item Esta função deve receber dois ponteiros constantes \code{c}{a} e \code{c}{b} do tipo 
            \code{c}{void *}

        \item O retorno deve ser um número inteiro que representa a relação entre os ponteiros:
            \begin{enumerate}
                \item zero, se \code{c}{a} e \code{c}{b} são iguais
                \item negativo, se \code{c}{a} é menor do que \code{c}{b}
                \item positivo, se \code{c}{a} é maior do que \code{c}{b}
            \end{enumerate}

        \item Como os parâmetros são ponteiros do tipo \code{c}{void *}, é preciso fazer a 
            coerção dos mesmos para o tipo apropriado na implementação

    \end{itemize}

\end{frame}

\begin{frame}[fragile]{Exemplo de uso da função \texttt{qsort()}}
    \inputsnippet{c}{1}{17}{qsort.c}
\end{frame}

\begin{frame}[fragile]{Exemplo de uso da função \texttt{qsort()}}
    \inputsnippet{c}{18}{37}{qsort.c}
\end{frame}
