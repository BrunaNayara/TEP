\section{Categorização de problemas}

\begin{frame}[fragile]{Categorias de problemas}

    \begin{itemize}
        \item Os eventos ACM ICPC tem, em geral, de 8 a 12 problemas, a serem resolvidos
            em 5 horas

        \item Os \textit{rounds} do Codeforces tem entre 4 e 7 problemas, a serem resolvidos 
            em 2 horas 

        \item Os problemas podem ser classificados em 10 Categorias
        \begin{center}
        \texttt{
        \begin{minipage}{0.45\textwidth}
            \begin{enumerate}
            \item \textit{Ad-hoc}
            \item Busca Completa
            \item Dividir e Conquistar
            \item Gulosos
            \item Programação Dinâmica
            \end{enumerate}
        \end{minipage}
        \begin{minipage}{0.45\textwidth}
            \begin{enumerate}
            \setcounter{enumi}{5}
            \item Grafos
            \item Matemática
            \item Strings
            \item Geometria Computacional
            \item Estruturas de Dados
            \end{enumerate}
        \end{minipage}
        }
        \end{center}
        \item Há problemas que não se enquadram nestas categorias, porém com uma frequência
            incomum de aparição

        \item As subcategorias que um competidor deve dominar são: ordenação, recursão e
            a própria linguagem de programação escolhida
    \end{itemize}

\end{frame}

\begin{frame}[fragile]{Classificação pessoal dos problemas}

    \begin{itemize}
        \item Um competidor pode classificar os problemas em 3 cateogorias
        \texttt{
        \begin{enumerate}[A.]
            \item Já resolvi um parecido e posso resolver de novo rapidamente
            \item Já vi um parecido e sei que não consigo resolver 
            \item Nunca vi
        \end{enumerate}
        }

        \item Para se tornar um competidor efetivo é preciso 
        \begin{enumerate}
                \item classificar a maioria dos problemas como \texttt{A}
                \item minimizar os problemas \texttt{B}
                \item usar os conhecimentos acumulados no treinamento para atacar os problemas \texttt{C}
        \end{enumerate}

        \item É importante manter um registro dos problemas do tipo \texttt{B} e \texttt{C} 
            encontrados, e tentar resolvê-los posteriormente, através da discussão com os outros
            competidores ou leitura dos editoriais e soluções disponíveis

        \item A resolução de problemas que antes eram inacessíveis promove o crescimento técnico
            e profissional do competidor
    \end{itemize}

\end{frame}

\begin{frame}[fragile]{Teste do código da solução}

    \begin{itemize}
        \item Os exemplos de entrada e saída do problema, em geral, são triviais

        \item No caso de uma solução \texttt{WA}, é preciso achar um teste que quebre a solução

        \item Para gerar estes testes, é preciso avaliar os \textit{corner cases}:
            \begin{enumerate} 
                \item os maiores valores possíveis para as entradas
                \item os menores valores possíveis para as entradas
                \item variáveis com valores zerados ou negativos
                \item grafos desconetados, polígonos não convexos, polígonos degenerados
                \item todos as entradas iguais
                \item entradas em ordem crescente ou decrescente
            \end{enumerate} 

        \item Também é bom verificar se a solução proposta está assumindo alguma característica
            da entrada que não foi descrita explicitamente no problema (ordenação, valores
            máximos, etc)
    \end{itemize}

\end{frame}
