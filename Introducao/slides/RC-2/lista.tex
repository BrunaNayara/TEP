\documentclass[12pt]{article}
\usepackage[brazil]{babel}
\usepackage[latin1]{inputenc}
\usepackage{graphicx}
\usepackage{geometry}
\usepackage[scaled=0.9]{helvet}
\usepackage{enumerate}
\usepackage{multicol}
\usepackage[table]{xcolor}
\usepackage{array}
\usepackage{xtab}
\usepackage{booktabs}

\setlength{\parskip}{\baselineskip}
\geometry{verbose,tmargin=1.5cm,bmargin=1.5cm,lmargin=1.5cm,rmargin=1.5cm}
\renewcommand{\rmdefault}{ptm}

\begin{document}

\begin{center}
	\includegraphics[scale=0.45]{fga.ps}
	% fga.ps: 1087x103 pixel, 72dpi, 38.35x3.63 cm, bb=0 0 1087 103
\end{center}


\vspace{0.1in}

\begin{center}
\begin{large}
\textsc{\textbf{Lista de Exerc�cios}}
\end{large}

{\bf Revis�o C/C++: Tipos de dados de usu�rio}
\end{center}

\vspace{0.2in}

\begin{enumerate}

	\item Execute o c\'odigo {\it cadastro.c} e justifique o tamanho obtido para a estrutura {\bf Cadastro}.

	\item Em rela��o ao c\'odigo {\it rotate.c}, 
	\begin {enumerate}
		\item Qual � o tamanho, em {\it bytes}, da uni�o {\bf word32}?

		\item Reescreva a fun��o {\it rotate}(), de modo que ela receba como par�metro a quantidade de {\it bytes} 
			a serem rotacionados e a dire��o da rota��o (esquerda ou direita).
		
		\item Escreva uma nova fun��o de rota��o que rotacione os {\it bits} de um inteiros, dadas a dire��o e o n�mero
			de posi��es a serem rotacionadas.

	\end{enumerate}

	\item Em rela��o aos c\'odigos {\it complex.h} e {\it complex.cpp}, 
	\begin {enumerate}
		\item Implemente as opera��es de subtra��o e divis�o de n�meros complexos.

		\item Implemente m�todos que retornem o raio e o �ngulo do n�mero complexo na forma polar.
	\end{enumerate}

\end{enumerate}

\end{document}
