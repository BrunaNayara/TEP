\section{Depuração}

\begin{frame}[fragile]{Depuração}

	\begin{itemize}
		\item A depuração (ou \textit{debug}) é o processo de remover ou reduzir o número de 
		erros (\textit{bugs}) em um programa

		\item Ela abrange desde impressões em pontos-chave do programa ou até programas que 
        acompanham a execução passo-a-passo (depuradores ou \textit{debuggers})

		\item Um depurador famoso e livre é o \texttt{gdb} (\textit{GNU Project Debugger})
		
		\item Para que o executável de um código gerado pelo \texttt{gcc} possa ser utilizado pelo 
        \texttt{gdb}, é necessário adicionar a \textit{flag} \texttt{\lq -g\rq} na 
		linha de compilação.
		
    \end{itemize}
\end{frame}

\begin{frame}[fragile]{Invocando o \texttt{gdb}}

	\begin{itemize}
        \item O \texttt{gdb} é um programa em linha de comando, embora existam outros programas
        que fornecem uma interface gráfica para os comandos do \texttt{gdb} 

		\item O \texttt{gdb} pode ser invocado através do comando
		\begin{description}
			\item \texttt{\$ gdb prog}
		\end{description}
		onde \texttt{prog} é o nome do executável a ser depurado

		\item Caso o executável tenha sido gerado por múltiplos arquivos, a opção 
        \texttt{\lq -d\rq} indica o diretório onde se encontram os arquivos-fontes.

        \item Um executável gerado com a opção de depuração contém informações extra que 
        permitem a depuração, de modo que seu tamanho é, consequentemente, maior do que 
        um executável gerado sem esta opção
	\end{itemize}

\end{frame}

\begin{frame}[fragile]{Comandos básicos do \texttt{gdb}}

	\begin{description}
		\item [\texttt{run}] inicia a {execução} do programa

		\item [\texttt{kill}] {interrompe} a execução, permitindo a reinicialização com um novo \textit{run}

		\item [\texttt{quit}] {encerra} o \texttt{gdb}

		\item [\texttt{help}] fornece {informações} sobre o \texttt{gdb} e seus comandos

		\item [\texttt{list}] {imprime} parte do código-fonte

		\item [\texttt{break}] insere um {ponto de parada} (\textit{breakpoint}) na linha indicada.
        Se houverem múltiplos arquivos, o argumento deve ser 
        \texttt{nome\_do\_arquivo:numero\_da\_linha}. Também pode ser indicado o nome de uma função

	\end{description}

\end{frame}

\begin{frame}[fragile]{Comandos básicos do \texttt{gdb}}

	\begin{description}
		\item [\texttt{continue}] {continua} a execução até o {próximo} \textit{breakpoint}

		\item [\texttt{next}] {avança} para a {próxima linha} do código-fonte

		\item [\texttt{step}] se houver uma {função} na linha atual, entra na função

		\item [\texttt{print}] {imprime} o valor da {variável} passada como parâmetro

		\item [\texttt{watch}] gera informações {adicionais} quando a variável passada como 
        parâmetro é {lida} ou {escrita}

		\item [\texttt{info break}] {lista} os \textit{breakpoints}
					
		\item [\texttt{info watch}] {lista} os \textit{watchpoints}

	\end{description}

\end{frame}

\begin{frame}[fragile]{Exemplo para teste de depuração}
    \inputcode{c}{primes.h}
\end{frame}

\begin{frame}[fragile]{Exemplo para teste de depuração}
    \inputcode{c}{primes.c}
\end{frame}

\begin{frame}[fragile]{Exemplo para teste de depuração}
    \inputcode{c}{main.c}
\end{frame}
