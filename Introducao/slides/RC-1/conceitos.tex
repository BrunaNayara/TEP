\section{Conceitos elementares}

\begin{frame}[fragile]{C}

	\begin{itemize}
	
		\item C é uma linguagem de programação estaticamente tipada, de forma livre, imperativa e 
        de propósito geral

		\item Foi criada por Dennis Ritchie em 1972 no laboratório da Bell para desenvolvimento em 
        sistemas operacionais Unix

		\item {\bf Leitura complementar}: RITCHIE, Dennis M. {\it The Development of the C 
        Language}. AT\&T Bell Laboratories.  Murray Hill, New Jersey. 1993.

	\end{itemize}

    O arquivo \file{source.c} pode ser compilado com o GCC usando a seguinte linha de comando:

    \begin{center}
        \texttt{\$ gcc -o prog -W -Wall -Wextra -pedantic -O2 source.c}
    \end{center}

\end{frame}

\begin{frame}[fragile]{C++}

	\begin{itemize}
		\item C++ é uma linguagem de programação estaticamente tipada, de forma livre, 
        multi-paradigma e de propósito geral

        \item Foi criada por Bjarne Stroustrup em 1979 no laboratório da Bell, inicialmente como 
        uma extensão da linguagem C

		\item {\bf Leitura complementar}: STROUSTRUP, Bjarne.  {\it A History of C++: 1979-1991}. 
        AT\&T Bell Laboratories.  Murray Hill, New Jersey. 1994.
	\end{itemize}

    O arquivo \file{source.cpp} pode ser compilado com o GCC usando a seguinte linha de comando:

    \begin{center}
        \texttt{\$ g++ -o prog -W -Wall -Wextra -std=c++17 -O2 source.cpp}
    \end{center}
\end{frame}

\begin{frame}[fragile]{Hello World!}

	\begin{itemize}

		\item A prática recorrente é apresentar uma linguagem através do programa
		mais elementar de todos: o \textit{Hello World!}

		\item O propósito deste programa é ilustrar o mecanismo de saída (\textit{output}) da 
        linguagem

		\item Uma das maneiras de se medir (de forma superficial) a complexidade (sintática) de uma 
        linguagem é contabilizar o número de linhas do \textit{Hello World!}

        \item \textit{Hello World!} em Python 3:

        \inputcode{python}{hello.py3}

	\end{itemize}
	
\end{frame}

\begin{frame}[fragile]{\textit{Hello World!} em C/C++}

    \inputcode{c}{hello.c}
    \inputcode{cpp}{hello.cpp}

\end{frame}

\begin{frame}[fragile]{Variáveis}
	
    \metroset{block=fill}
    \begin{block}{Sintaxe para declaração de variaveis}
    \inputsyntax{c}{variaveis.st}
    \end{block}
	\begin{itemize}
        \item Variáveis são valores que podem ser modificados ao longo da execução do programa
        \item Em C, a cada variável é associado um dos 5 tipos primitivos de dados:
        \code{c}{char, short, int, double, float}

		\item Palavras-chave associadas ao:
		\begin{itemize}
			\item armazenamento: \code{c}{extern, static, register, auto}
			\item acesso:  \code{c}{const, volatile}
			\item modificador:  \code{c}{signed, unsigned,  long,  short}
		\end{itemize}
		
	\end{itemize}
	
\end{frame}

\begin{frame}[fragile]{Classificação das variáveis}

	\begin{small}
    \begin{center}
        \begin{tabularx}{\textwidth}{lp{1.5cm}X}
            \toprule
            \textbf{Aspecto} & \textbf{Classificação} & \textbf{Descrição} \\
            \midrule 
            \multirow{3}{*}{\hfill Localização} & globais & Podem ser acessadas em qualquer função \\
            & \cellcolor[gray]{0.9}locais & \cellcolor[gray]{0.9}Podem ser acessadas apenas no bloco em que foram declaradas \\
            & parâmetros de função & Podem ser acessados apenas pela função. São preenchidos na 
            chamada da função \\
            \midrule 
            \multirow{3}{*}{\hfill Acesso} & \cellcolor[gray]{0.9}gerais & L\cellcolor[gray]{0.9}eitura e escrita \\
            & constantes & Apenas leitura. Devem ser declaradas com um valor inicial \\
            & \cellcolor[gray]{0.9}voláteis & \cellcolor[gray]{0.9}Podem ser modificadas por programas externos \\
            \midrule
            \multirow{3}{*}{\hfill Armazenamento} & externas & A definição ocorre em outro arquivo \\
            & \cellcolor[gray]{0.9}estáticas & \cellcolor[gray]{0.9}O armazenamento é alocado uma única vez \\
            & registradores & O armazenamento é feito em registradores, não em memória \\
            \bottomrule
        \end{tabularx}
    \end{center}
	\end{small}

\end{frame}

\begin{frame}[fragile]{Exemplo de variáveis}
    \inputcode{c}{media_aritmetica.c}
\end{frame}
