\section{Macros}

\begin{frame}[fragile]{Dica \#9: Macro para laços \texttt{for}}

    \begin{itemize}
        \item O uso de macros pode encurtar o tamanho de padrões comuns em códigos C/C++

        \item Um destes padrões é o laço for que faz a travessia de um vetor de $N$ elementos

        \item Além da redução do tamanho, a macro tem a vantagem de evitar erros no incremento

            \inputcode{c++}{mfor.cpp}
    \end{itemize}

\end{frame}

\begin{frame}[fragile]{Dica \#10: Macros para \texttt{debug}}

    \begin{itemize}
        \item Macros podem ser utilizadas para imprimir informações de \textit{debug} do código

        \item Escrevendo as informações de saída em \code{c++}{cerr} a solução poderá ser
            aceita mesmo com os \textit{logs} ativados (desde que a escrita destes \textit{logs}
            não gere um TLE)

        \item Os juízes online definem a macro \code{c++}{ONLINE_JUDGE}, que pode ser usada para
            desabilitar os \textit{logs}

        \item As macros de \textit{log} devem ser simples de usar e criadas para os tipos mais
            comuns (variáveis de tipo primitivo, vetores, \textit{arrays}, etc)
    \end{itemize}

\end{frame}

\begin{frame}[fragile]{Exemplo de uso de macros de \texttt{log}}
    \inputcode{c++}{logs.cpp}
\end{frame}
