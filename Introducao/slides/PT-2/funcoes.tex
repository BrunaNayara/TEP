\section{Ponteiros e funções}

\begin{frame}[fragile]{Passagem de parâmetros}

	\begin{itemize}
	
		\item Os parâmetros das funções e métodos podem ser passados de duas maneiras: 
        por valor ou por referência

		\item Na passagem por valor, os valores das variáveis passadas para a função ou método são 
        copiados para os parâmetros formais

		\item Na passagem por referência, as funções recebem em seus parâmetros referências para 
        as variáveis passadas como parâmetros

        \item A passagem por cópia tem maior custo de execução do que a passagem por referência,
        mas garante que as variáveis passadas como parâmetros não serão modificadas pela
        função
	\end{itemize} 
 
\end{frame}
 
\begin{frame}[fragile]{Passagem de parâmetros em C/C++}

	\begin{itemize}
		\item Em C, a única forma de passagem de parâmetros é por valor 
    
        \item C++ suporta ambos métodos de passagem

		\item Para se utilizar a passagem por referência em C, deve-se declarar o parâmetro como um 
        ponteiro para o tipo de dado desejado

        \item Efetivamente, a "passagem por referência" do C é uma passagem por cópia, onde o
        ponteiro é copiado e o acesso à variável é feita através do ponteiro

		\item Embora o mesmo seja verdadeiro para C++, a linguagem suporta passagem por referência 
        nativamente

		\item O mecanismo de referência de C++ simplifica a notação, mas deixa apenas implícito 
        que os valores podem ser modificados pela função
	\end{itemize}

\end{frame}

\begin{frame}[fragile]{Exemplo de passagem de parâmetros em C}
    \inputsnippet{c}{1}{21}{referenciavsvalor.c}
\end{frame}

\begin{frame}[fragile]{Exemplo de passagem de parâmetros em C}
    \inputsnippet{c}{22}{41}{referenciavsvalor.c}
\end{frame}

\begin{frame}[fragile]{Exemplo de passagem de parâmetros em C++}
    \inputsnippet{cpp}{1}{21}{swap.cpp}
\end{frame}

\begin{frame}[fragile]{Exemplo de passagem de parâmetros em C++}
    \inputsnippet{cpp}{23}{39}{swap.cpp}
\end{frame}

\begin{frame}[fragile]{Ponteiros para funções}

    \metroset{block=fill}
    \begin{block}{Sintaxe para declaração de ponteiros para funções}
		\inputsyntax{c}{pf.st}
    \end{block}

	\begin{itemize}
		\item O ponto de entrada de uma função ocupa uma posição na memória: logo podemos declarar 
        ponteiros para funções

		\item O endereço de uma função pode ser obtido através do nome da função, como no caso de 
		vetores

		\item O ponteiro para função pode ser usado para fazer uma chamada a função, com a mesma 
        sintaxe de chamadas comuns. Exemplo:
            \inputsyntax{c}{ex.st}
		\item Ponteiros para funções permitem a construção de programas dinâmicos e podem aumentar 
        a organização do código
	\end{itemize}

\end{frame}

\begin{frame}[fragile]{Exemplo de uso de ponteiros de funções}
    \inputcode{c}{plugin.h}
\end{frame}

\begin{frame}[fragile]{Exemplo de uso de ponteiros de funções}
    \inputsnippet{c}{1}{21}{plugin.c}
\end{frame}

\begin{frame}[fragile]{Exemplo de uso de ponteiros de funções}
    \inputsnippet{c}{23}{43}{plugin.c}
\end{frame}

\begin{frame}[fragile]{Exemplo de uso de ponteiros de funções}
    \inputsnippet{c}{45}{54}{plugin.c}
\end{frame}

\begin{frame}[fragile]{Exemplo de uso de ponteiros de funções}
    \inputcode{c}{soma.c}
\end{frame}

\begin{frame}[fragile]{Exemplo de uso de ponteiros de funções}
    \inputcode{c}{subtracao.c}
\end{frame}

\begin{frame}[fragile]{Exemplo de uso de ponteiros de funções}
    \inputsnippet{c}{1}{21}{calculadora.c}
\end{frame}

\begin{frame}[fragile]{Exemplo de uso de ponteiros de funções}
    \inputsnippet{c}{23}{43}{calculadora.c}
\end{frame}

\begin{frame}[fragile]{Exemplo de uso de ponteiros de funções}
    \inputsnippet{c}{45}{62}{calculadora.c}
\end{frame}

\begin{frame}[fragile]{Exemplo de uso de ponteiros de funções}
    \inputsnippet{c}{63}{83}{calculadora.c}
\end{frame}

\begin{frame}[fragile]{Exemplo de uso de ponteiros de funções}
    \inputsnippet{c}{84}{102}{calculadora.c}
\end{frame}
